\documentclass{beamer}
\usetheme{metropolis}
%\setsansfont[BoldFont={Fira Sans SemiBold}]{Fira Sans Book}
%\setsansfont{Fontin}
%\setsansfont{Gillius ADF No2}
%\setsansfont{Phetsarath OT}
\setsansfont{Source Sans Pro}
\setmonofont{Source Code Pro}

\hypersetup{colorlinks=true,
            linkcolor=mRustLightOrange,
            menucolor=mRustLightOrange,
            pagecolor=mRustLightOrange,
            urlcolor=mRustLightOrange}
\usepackage{csquotes}
\usepackage{comment}
\usepackage{xcolor}
\usepackage{minted}
\usepackage{mathtools}

\newfontfamily\codefont{Source Code Pro}
\newcommand\code[1]{\,{\color[HTML]{884400}#1}\,}
\newcommand\source[1]{$\rightarrow$ via #1}

\title{Concurrency}
\date{\today}
\author{Lukas Prokop}
\institute{RustGraz community\vfill\hfill\includegraphics[height=2cm]{images/rustacean-orig-noshadow.png}}
\begin{document}
\maketitle

\section{Prologue}

\begin{frame}[fragile]{Recap: Rusty days}
  \begin{center}
    \includegraphics[width=.9\textwidth]{images/rdconf.jpeg}
  \end{center}
  \begin{itemize}
    \item Webference organized by \href{https://rusty-days.org/}{Rust Wrocław}
    \item 8 talks between 2020-07-27 and 2020-07-31
    \item I will give a short talk summary
  \end{itemize}
\end{frame}

\begin{frame}[standout]
  Mon, 2020-07-27 --- Steve Klabnik \\
  \enquote{Should we have a Rust 2021 edition?}
\end{frame}

\begin{frame}[fragile]{Recap: Rust 2021 edition Talk}
rustc is a multi-pass compiler: AST → HIR → MIR → LLVM-IR
  \begin{description}
    \item[HIR] expand source code to simpler primitive statements (type checking, method lookup)
    \item[MIR] is all about control flow (control-flow, borrow checking)
  \end{description}
  \begin{itemize}
    \item \emph{lowering}: \enquote{The MIR is then lowered to LLVM-IR}
    \item \enquote{cargo check} omits code generation step
    \item \emph{goal}: \enquote{query-based} compiler (memoized answers)
    \item editions are not allowed to differ in MIR
  \end{itemize}
  \begin{description}
    \item[nightly release] every night
    \item[stable + beta release] every six weeks
  \end{description}
\end{frame}

\begin{frame}[fragile]{Recap: Rust 2021 edition Talk}
  \textbf{Editions:}
  \begin{description}
    \item[is] longer-term progress, breaking changes, new editions are opt-in, can't change everything
    \item[can] can {introduce new keywords, repurpose syntax (e.g. deprecate trait, introduce dyn trait)}
    \item[cannot] cannot {change coherence rules, standard library}
  \end{description}
  \begin{itemize}
    \item when do editions happen? no policy
    \item edition 2018 (= rust 1.31) was \enquote{a really big project for the various teams} and \enquote{a lot of burnout amongst contributors} (\enquote{I was a total mess})
    \item editions feature-driven or time-boxed? time-boxed! 3 years is a nice compromise between yearly and 5-year releases
    \item Steve: We should have a 2021 edition. Much smaller than 2018.
  \end{itemize}
\end{frame}

\begin{frame}[fragile]{Recap: Rust 2021 edition Talk}
  \begin{minted}[fontsize=\small]{text}
meisterluk@gardner ~ % cat test.rs

fn async() -> u64 {
    42
}

fn main() {
    println!("{}", async());
}

meisterluk@gardner ~ % rustc test.rs
  \end{minted}
\end{frame}

\begin{frame}[fragile]{Recap: Rusty days}
  \begin{minted}[fontsize=\small]{text}
meisterluk@gardner ~ % rustc --edition=2018 test.rs

error: expected identifier, found keyword `async`
 --> test.rs:1:4
  |
1 | fn async() -> u64 {
  |    ^^^^^ expected identifier, found keyword
  |
help: you can escape reserved keywords
      to use them as identifiers
  |
1 | fn r#async() -> u64 {
  |    ^^^^^^^
  \end{minted}
\end{frame}

\begin{frame}[standout]
  Mon, 2020-07-27 --- Michalina Kotwica \\
  \enquote{Low-level optimization of algebraic and similar structures}
\end{frame}

\begin{frame}[fragile]{Recap: Low-level optimization Talk}
  \begin{itemize}
    \item type algebra
    \item unit \& never type
    \item memory layout (ABI) of composite types (struct, enum, futures)
    \item struct Bar(u16, u8, u16, u8): \\
          \texttt{C:    11 11 33 xx 22 22 44 xx}
          \texttt{Rust: 11 11 22 22 33 44}  (fields reordered)
    \item \mintinline{rust}{Option} uses a discriminator, \mintinline{rust}{Option<Option<u64>>} uses only one discriminator
    \item 41.2~\% of enums have no type arguments, 16.4~\% of enums have one type argument, 23.7~\% of enums have two type arguments
  \end{itemize}
\end{frame}

\begin{frame}[standout]
  Tue, 2020-07-28 --- Peter Parkanyi \\
  \enquote{Fast encrypted backups with Rust - \enquote{How I stopped worrying and love mmap}}
\end{frame}

\begin{frame}[fragile]{Recap: Fast encrypted backups Talk}
  \url{https://github.com/rsdy/zerostash}
  \begin{itemize}
    \item End-to-End Encryption
    \item latency and throughput
    \item Zero-metadata data stash: deduplicated, works as a file system and key/value store
    \item Cryptographic primitives:
      \begin{description}
        \item[Passwords] Argon2
        \item[Indexing] Blake2
        \item[Compression] LZ4
        \item[Encryption] ChaCha20-Poly1305
        \item[Deduplication] SeaHash
      \end{description}
    \item Profiling: perf on Linux, Instruments on macOS
    \item mmap versus read
  \end{itemize}
\end{frame}

\begin{frame}[standout]
  Wed, 2020-07-29 --- Lachezar Lechev \\
  \enquote{Drone Control - \enquote{Controlling a drone using Rust over WiFi}}
\end{frame}

\begin{frame}[fragile]{Recap: Drone Control Talk}
  \url{https://github.com/AeroRust/Welcome}
  \begin{itemize}
    \item TCP connections {Handshake, establish connection}
    \item JSON requests \& responses
    \item ping-pong within 7s
    \item \href{https://crates.io/crates/scroll}{scroll} crate: \enquote{A suite of powerful, extensible, generic, endian-aware Read/Write traits for byte buffers}
  \end{itemize}
\end{frame}

\begin{frame}[standout]
  Wed, 2020-07-29 --- Nell Shamrell - Harrington \\
  \enquote{The Rust Borrow Checker - A Deep Dive}
\end{frame}

\begin{frame}[fragile]{Recap: Drone Control Talk}
  \href{https://www.slideshare.net/NellShamrell/the-rust-borrow-checker}{Slides on slideshare}
  \begin{itemize}
    \item \enquote{Is the Borrow Checker a friend or a foe?}
    \item Stages of Compilation:
      \begin{enumerate}
        \item Lexical Analysis
        \item Parsing
        \item Semantic Analysis (Borrow Checker!)
        \item Optimization
        \item Code Generation
      \end{enumerate}
    \item BC tracks initializations/moves and applies lifetime inference
    \item Lifetime of a variable has two definitions
      \begin{itemize}
        \item \enquote{Span of time before the value of a variable gets freed}
        \item \enquote{scope of a variable}
      \end{itemize}
    \item \enquote{If you make a reference to a value, the lifetime of that reference cannot outlive the scope of the value}
    \item \url{https://rustc-dev-guide.rust-lang.org/}
  \end{itemize}
\end{frame}

\begin{frame}[standout]
  Thu, 2020-07-30 - Jan-Erik Rediger \\
  \enquote{Leveraging Rust to build cross-platform mobile libraries}
\end{frame}

\begin{frame}[fragile]{Recap: Mobile Libraries Talk}
  \href{https://www.slideshare.net/NellShamrell/the-rust-borrow-checker}{Slides on slideshare}
  \begin{itemize}
    \item \href{https://telemetry.mozilla.org/}{Firefox Telemetry project}
    \item Collect performance metrics for our products, package pings at controlled schedules
    \item Three Principles
      \begin{itemize}
        \item Stay Lean
        \item Build Security
        \item Engage Your Users
      \end{itemize}
    \item Telemetry \emph{scalars}: \texttt{Scalars.yaml} (metadata: bug\_numbers, description, expires, notification\_emails, \dots)
  \end{itemize}
\end{frame}

\begin{frame}[fragile]{Recap: Mobile Libraries Talk}
  \begin{itemize}
    \item Glean core → Glean FFI → Glean Kotlin/Swift → Android/iOS app
    \item \href{https://crates.io/crates/cbindgen}{cbindgen} crate: \enquote{A tool for generating C bindings to Rust code.}
    \item 10 Glean compilation targets supported
    \item Hello World with JNI (also see Otavio Pace's talk)
    \item tagged unions are generated by bindgen for rust enums
    \item ProtoBuf to serialize data
    \item R8 minifies/optimizes JVM bytecode (successor to proguard)
    \item here: only invoke Rust code from Kotlin (never the other way around)
  \end{itemize}
\end{frame}

\begin{frame}[standout]
  Fri, 2020-07-31 - Luca Palmieri \\
  \enquote{Are we observable yet? Telemetry for Rust APIs - metrics, logging, distributed tracing}
\end{frame}

\begin{frame}[fragile]{Recap: Telemetry for Rust APIs Talk}
  \begin{itemize}
    \item Developer of \href{https://www.donatedirect.org.au/who-we-are.html}{DonateDirect}
    \item New project: fast versus reliable (metrics, tracing, logs, \dots)
    \item Claim: convenience beats correctness
    \item Metrics give us an aggregate picture of the system state
    \item \href{https://crates.io/crates/actix-web-prom}{actix\_web\_prom} crate provides a pluggable middleware with standard Prometheus metrics out of the box
    \item \href{https://docs.rs/log/0.4.6/log/}{log} crate and \href{https://crates.io/crates/tracing}{tracing} crate to dump structured logging with JSON output and forward spans to \href{https://www.elastic.co/what-is/elasticsearch}{elasticsearch} and then \href{https://www.elastic.co/kibana}{kibana}
  \end{itemize}
\end{frame}

\begin{frame}[fragile]{Recap: Telemetry for Rust APIs Talk}
  \textbf{Key takeaways:}
  \begin{itemize}
  \item Lack of telemetry is a ticking bomb
  \item Diagnostic instrumentation has to be easy
  \item Metrics to alert and monitor system state
  \item High cardinality is key to being able to detect and triage unknown unknowns
  \item Span as unit of work abstraction
  \item You must be able to trace a request across different services
  \end{itemize}
\end{frame}

\begin{frame}[standout]
  Fri, 2020-07-31 --- Tim McNamara \\
  \enquote{How 10 open source projects manage unsafe code}
\end{frame}

\begin{frame}[fragile]{Recap: Managing unsafe code Talk}
  \textbf{Unsafe guidelines for the impatient:}
  \begin{itemize}
    \item Use \mintinline{rust}{#[deny(unsafe_code)]}
    \item Add comments to all unsafe blocks
    \item Let someone read the unsafe block comment. \\
          If they cannot explain afterwards, revise the comment
  \end{itemize}

  \textbf{Remarks:}
  \begin{itemize}
    \item Question \texttt{unsafe}! Look for safe alternatives
    \item rust std: \enquote{Unsafe code block need a comment explaining why they're ok}
    \item \enquote{We try to create a situation where we, as a team, are building safe software and we are mentally switched on if we go to the unsafe module}
    \item \href{https://github.com/rust-lang/unsafe-code-guidelines}{UCG WG - Rust's Unsafe Code Guidelines Working Group}
  \end{itemize}
\end{frame}

\begin{frame}[fragile]{Recap: Managing unsafe code Talk}
  \begin{itemize}
    \item \texttt{actix\_web} \href{https://github.com/rust-user-group-graz/07-unsafe}{incident}
    \item \mintinline{rust}{#[deny(unsafe_code)]} and \mintinline{rust}{#[allow(unsafe_code)]} (can be nested)
    \item \href{https://crates.io/crates/cargo-geiger}{cargo-geiger} \enquote{detects usage of unsafe Rust in a Rust crate and its deps}, introduces \mintinline{rust}{#![forbid(unsafe_code)]}
    \item \href{https://crates.io/crates/exa}{exa} uses syscalls natively, \href{https://crates.io/crates/blake3}{BLAKE3} uses SIMD instructions, \href{https://crates.io/crates/firecracker}{Firecracker} interacts with a hybervisor, \href{https://gitlab.gnome.org/GNOME/librsvg}{librsvg} talks to GLib, \href{https://crates.io/crates/toolshed}{toolshed} deals with pointers in one module, \href{https://terminusdb.com/}{terminusdb} interacts with Prolog, \href{https://fuchsia.dev/}{Fuchsia OS} interacts with the non-rust kernel
  \end{itemize}
\end{frame}



\section{Dialogue}

\begin{frame}[standout]
  Concurrency
\end{frame}

\begin{frame}[fragile]{Definition: concurrency}
  \begin{quote}
    In computer science, \emph{concurrency} is the ability of different parts or units of a program, algorithm, or problem to be executed out-of-order or in partial order, without affecting the final outcome. \\
    ---\href{https://en.wikipedia.org/w/index.php?title=Concurrency_(computer_science)&oldid=971297468}{Concurrency (computer science)}
  \end{quote}
\end{frame}

\begin{frame}[fragile]{Definition: parallelism}
  \begin{quote}
    \emph{Parallel computing} is a type of computation in which many calculations or the execution of processes are carried out simultaneously. \\
    ---\href{https://en.wikipedia.org/w/index.php?title=Parallel_computing&oldid=961837202}{Parallel computing}
  \end{quote}
\end{frame}

\begin{frame}[fragile]{Models of concurrency}
  Formally, different models exist:
  \begin{itemize}
    \item Parallel random-access machine
    \item Actor model
    \item Petri nets
    \item Process calculi (CCS, CSP, $\pi$-calculus)
    \item Tuple spaces
    \item Simple Concurrent Object-Oriented Programming (CSOOP)
    \item Reo Coordination Language
  \end{itemize}
  \dots\ to design distributed systems.
\end{frame}

\begin{frame}[fragile]{Models of concurrency}
  But, in rust, we stick to a von-Neumann-like model:
  \begin{itemize}
    \item Instructions are executed in order.
    \item We have stacks, heaps, bss and data sections to organize the memory.
    \item We make syscalls to the kernel and compile against some ISA.
    \item There is \textbf{no} runtime compiled into every executable (improves C interoperability)
  \end{itemize}
  Thus it is a question of an API. We define \emph{concurrent units} and they might run in \emph{parallel}.
\end{frame}

\begin{frame}[fragile]{Definition: concurrency}
  Concurrency as a question of level of granularity:
  \begin{enumerate}
    \item Instruction
    \item Blocks of instructions
    \item Function
    \item Several stacks, one heap
    \item Process
  \end{enumerate}
  In general: performance as incentive.
\end{frame}

\begin{frame}[fragile]{Problems of concurrency}
  \begin{enumerate}
    \item Data dependency (synchronization problem)
      \begin{description}
        \item[let] A and B be two concurrent units. \\
          Both want to increment $x$
        \item[A] reads that $x$ is $41$
        \item[B] reads that $x$ is $41$
        \item[A] increments $x$ and gets $y \coloneqq 42$
        \item[B] increments $x$ and gets $z \coloneqq 42$
        \item[A] writes $x \coloneqq y$
        \item[B] writes $x \coloneqq z$
        \item[$\mathbf x$] is 42
      \end{description}
    \item Concurrent units need to exchange messages
    \item Concurrent unit waits for an event
    \item Interrupt concurrent unit (\emph{preemption})
    \item Determine concurrent unit has finished
  \end{enumerate}
\end{frame}

\begin{frame}[standout]
  In rust (from low-level to high-level)
\end{frame}

\begin{frame}[standout]
  On instruction-level
\end{frame}

\begin{frame}[fragile]{SIMD instructions}
  SIMD = \textbf{s}ingle \textbf{i}nstruction, \textbf{m}ultiple \textbf{d}ata

  Run one instruction, apply arithmetic/logic/data-handling/memory instructions to several values simultaneously. 8/16/32/64/128/512

  \begin{description}
    \item[RISC-V]
      I prefer to talk about the RISC-V ISA, but RISC-V basically dropped its \href{https://en.wikipedia.org/w/index.php?title=RISC-V&oldid=970523444#Packed_SIMD}{Packed SIMD} extension and develops a new \href{https://riscv.org/specifications/isa-spec-pdf/}{\enquote{P} extension}. Thus, I switch to x86\_64.
    \item[x86\_64]
      \href{https://en.wikipedia.org/w/index.php?title=Streaming_SIMD_Extensions&oldid=960385254}{Streaming SIMD Extensions (SSE)} and \href{https://en.wikipedia.org/w/index.php?title=Advanced_Vector_Extensions&oldid=971220867}{Advanced Vector Extensions (AVX)}
  \end{description}
\end{frame}

\begin{frame}[fragile]{SSE and AVX}
  Operands (put into XMM/YMM registers):
  \begin{description}
    \item[SSE] four f32
    \item[SSE2] two f64, two i64, four i32, eight i16, sixteen u8
    \item[SSE3] (only 13 new instructions)
    \item[SSE4] (only 54 new instructions)
    \item[AVX] eight f32, four f64
    \item[AVX2] 256-bit registers for almost everything
    \item[AVX-512] 512-bit registers but instructions split into multiple extensions
  \end{description}
\end{frame}

\begin{frame}[fragile]{SIMD support}
  \begin{minted}[fontsize=\small]{text}
% cat /proc/cpuinfo | grep "sse\|avx"                                :(
flags   : fpu … sse sse2 … ssse3 … sse4_1 sse4_2
          … avx … avx2 … flush_l1d
flags   : fpu … sse sse2 … ssse3 … sse4_1 sse4_2
          … avx … avx2 … flush_l1d
flags   : fpu … sse sse2 … ssse3 … sse4_1 sse4_2
          … avx … avx2 … flush_l1d
flags   : fpu … sse sse2 … ssse3 … sse4_1 sse4_2
          … avx … avx2 … flush_l1d
  \end{minted}
\end{frame}

\begin{frame}[fragile]{(No) SIMD instructions}
  \begin{minted}{rust}
fn foo(a: &[u8], b: &[u8], c: &mut [u8]) {
    for ((a, b), c) in a.iter().zip(b).zip(c) {
      println!("{} {} {}", a, b, c);
        *c = *a + *b;
    }
}

fn main() {
  let a: [u8; 4] = [0xDE, 0xAD, 0xBE, 0xEF];
  let b: [u8; 4] = [0x00, 0x01, 0x02, 0x03];
  let mut c = [0u8; 4];
  foo(&a, &b, &mut c);

  println!("{:?}", c);
}

  \end{minted}
\end{frame}

\begin{frame}[fragile]{(No) SIMD instructions}
  On \href{https://godbolt.org/z/678vhz}{godbolt} with \texttt{-C opt-level=1} (or 0):

  \begin{minted}{nasm}
.LBB58_2:
; …
mov     rdx, qword ptr [rsp + 24]
movzx   ecx, byte ptr [rcx]
add     cl, byte ptr [rax]
mov     byte ptr [rdx], cl
mov     rdi, r14
; …
jne     .LBB58_2
  \end{minted}
\end{frame}

\begin{frame}[fragile]{SIMD instructions}
  On \href{https://godbolt.org/z/eKabj8}{godbolt} with \texttt{-C opt-level=2} (or higher):

  \begin{minted}{nasm}
movdqu  xmm2, xmmword ptr [rdx + rcx + 32]
paddb   xmm2, xmm0
movdqu  xmm0, xmmword ptr [rdx + rcx + 48]
paddb   xmm0, xmm1
movdqu  xmmword ptr [r8 + rcx + 32], xmm2
  \end{minted}

  \href{https://www.felixcloutier.com/x86/paddb:paddw:paddd:paddq}{paddb}: \enquote{add packed integer} instruction
\end{frame}

\begin{frame}[fragile]{SIMD instructions}
  via \href{https://doc.rust-lang.org/1.29.0/std/arch/}{std::arch} (\enquote{SIMD and vendor intrinsics module}):

  \begin{minted}[fontsize=\small]{rust}
#[cfg(all(
  any(target_arch = "x86", target_arch = "x86_64"),
  target_feature = "avx2"
))]
fn foo() {
  #[cfg(target_arch = "x86")]
  use std::arch::x86::_mm256_add_epi64;
  #[cfg(target_arch = "x86_64")]
  use std::arch::x86_64::_mm256_add_epi64;

  unsafe {
    _mm256_add_epi64(...);
  }
}
  \end{minted}
\end{frame}

\begin{frame}[fragile]{SIMD summary}
  \begin{itemize}
    \item rust uses the LLVM stack, which implements \emph{auto-vectorization}
    \item usually, we get SIMD instructions for free
    \item sometimes, you want to use the explicitly
    \item you can always use inline assembly with unsafe (\texttt{\href{https://docs.rs/x86/0.29.0/x86/time/fn.rdtsc.html}{x86::time::rdtsc}})
    \item \href{https://github.com/rust-lang/packed_simd}{packed\_simd} provides a high-level API
    \item also: \href{https://crates.io/crates/faster}{faster}, \href{https://crates.io/crates/ssimd}{ssimd}
    \item mostly, libraries are specific for an application domain, e.g.:
    \begin{itemize}
      \item \href{https://github.com/novacrazy/numeric-array}{numeric-array} crate
      \item \href{https://crates.io/crates/directx_math}{directx\_math} crate
      \item \href{https://crates.io/crates/pqcrypto-classicmceliece}{pqcrypto-classicmceliece} crate
    \end{itemize}
  \end{itemize}
  % http://huonw.github.io/blog/2015/08/simd-in-rust/ (2015 though!)
\end{frame}

\begin{frame}[fragile]{Atomic instructions}
  All atomic types in this module are guaranteed to be lock-free if they're available.

  \vspace{15pt}
  \begin{minipage}[l]{0.45\textwidth}
    \begin{itemize}
      \item AtomicBool
      \item AtomicI8
      \item AtomicI16
      \item AtomicI32
      \item AtomicI64
      \item AtomicIsize
    \end{itemize}
  \end{minipage}
  \begin{minipage}[l]{0.45\textwidth}
    \begin{itemize}
      \item AtomicPtr
      \item AtomicU8
      \item AtomicU16
      \item AtomicU32
      \item AtomicU64
      \item AtomicUsize
    \end{itemize}
  \end{minipage}
\end{frame}

\begin{frame}[fragile]{AtomicU64 example}
  \begin{minted}[fontsize=\small]{rust}
use std::sync::atomic::{AtomicU64, Ordering};
  
fn main() {
    let uint = AtomicU64::new(41);
    uint.store(42, Ordering::SeqCst);
    println!("{}", uint.load(Ordering::SeqCst));
}
  \end{minted}

  Memory orderings?

  \begin{minted}[fontsize=\small]{rust}
pub enum Ordering {
  Relaxed, Release, Acquire,
  AcqRel, SeqCst,
}
  \end{minted}

  Rust's memory orderings are the same as those of C++20.
\end{frame}

\begin{frame}[standout]
  On blocks of instructions
\end{frame}

\begin{frame}[fragile]{Mutex}
  \begin{itemize}
    \item \enquote{A mutual exclusion primitive useful for protecting shared data}
    \item A mutex can be \mintinline{rust}{lock}ed (true/blocked) or released
    \item Also \mintinline{rust}{try_lock} (true/false)
    \item Only one concurrent unit can hold a lock simultaneously
    \item If the mutex lock goes out of scope, another unit can acquire the lock.
  \end{itemize}
\end{frame}

\begin{frame}[fragile]{Mutex}
  \begin{minted}{rust}
use std::sync::Mutex;
  
fn main() {
    let val = Mutex::new(0u8);
    {
        *val.lock().unwrap() += 1;
        *val.lock().unwrap() += 1;
        *val.lock().unwrap() += 1;
    }
    println!("{}", val.lock().unwrap());
}
  \end{minted}
\end{frame}

\begin{frame}[fragile]{Mutex}
  \begin{minted}{rust}
use std::sync::{Arc, Mutex};
use std::thread;

fn main() {
    let mutex = Arc::new(Mutex::new(0));
    let c_mutex = mutex.clone();

    thread::spawn(move || {
        *c_mutex.lock().unwrap() = 10;
    }).join().expect("thread::spawn failed");
    println!("{}", *mutex.lock().unwrap());
}
  \end{minted}
\end{frame}

\begin{frame}[fragile]{Mutex poisoning}
  \begin{quote}
The mutexes in this module implement a strategy called \emph{poisoning} where a mutex is considered poisoned whenever a thread panics while holding the mutex. Once a mutex is poisoned, all other threads are unable to access the data by default as it is likely tainted (some invariant is not being upheld). \\
    ---Struct \href{https://doc.rust-lang.org/std/sync/struct.Mutex.html}{\mintinline{rust}{std::sync::Mutex}}
  \end{quote}
\end{frame}

\begin{frame}[fragile]{Mutex poisoning}
  A poisoned mutex, however, does not prevent all access to the underlying data.

  \begin{minted}{rust}
let mut guard = match lock.lock() {
    Ok(guard) => guard,
    Err(poisoned) => poisoned.into_inner(),
};

*guard += 1;
  \end{minted}
\end{frame}

\begin{frame}[fragile]{std::sync}
  \mintinline{rust}{std::sync} members:

  \begin{itemize}
    \item \href{https://doc.rust-lang.org/std/sync/struct.Arc.html}{Arc}: A thread-safe reference-counting pointer. 'Arc' stands for 'Atomically Reference Counted'.
    \item \href{https://doc.rust-lang.org/std/sync/struct.Barrier.html}{Barrier}: A barrier enables multiple threads to synchronize the beginning of some computation.
    \item \href{https://doc.rust-lang.org/std/sync/struct.Condvar.html}{CondVar}: A Condition Variable
    \item \href{https://doc.rust-lang.org/std/sync/mpsc/index.html}{mpsc}: Multi-producer, single-consumer FIFO queue communication primitives
    \item \href{https://doc.rust-lang.org/std/sync/struct.Once.html}{Once}: A synchronization primitive which can be used to run a one-time global initialization
    \item \href{https://doc.rust-lang.org/std/sync/struct.RwLock.html}{RwLock:} A reader-writer lock (number of readers or at most one writer at any point in time)
  \end{itemize}
\end{frame}

\begin{frame}[fragile]{Barrier example}
  \begin{minted}[fontsize=\scriptsize]{rust}
use std::sync::{Arc, Barrier};
use std::thread;

let mut handles = Vec::with_capacity(10);
let barrier = Arc::new(Barrier::new(10));
for _ in 0..10 {
    let c = barrier.clone();
    // The same messages will be printed together.
    // You will NOT see any interleaving.
    handles.push(thread::spawn(move|| {
        println!("before wait");
        c.wait();
        println!("after wait");
    }));
}
// Wait for other threads to finish.
for handle in handles {
    handle.join().unwrap();
}
  \end{minted}
\end{frame}

\begin{frame}[fragile]{mpsc channels}
  \begin{minted}[fontsize=\scriptsize]{rust}
use std::thread;
use std::sync::mpsc::channel;

// Create a shared channel that can be sent along
// from many threads where tx is the sending half
// (tx for transmission), and rx is the receiving
// half (rx for receiving).
let (tx, rx) = channel();
for i in 0..10 {
    let tx = tx.clone();
    thread::spawn(move|| {
        tx.send(i).unwrap();
    });
}

for _ in 0..10 {
    let j = rx.recv().unwrap();
    assert!(0 <= j && j < 10);
}
  \end{minted}
\end{frame}

\begin{frame}[standout]
  On OS thread level
\end{frame}

\begin{frame}[fragile]{threads}
  \begin{minted}[fontsize=\scriptsize]{rust}
use std::thread;

fn main() {
  let handler = thread::spawn(|| {
    // thread code
  });

  handler.join().unwrap();
}
  \end{minted}

  Copy stack (expensive?), share heap.
\end{frame}

\begin{frame}[fragile]{threads}
  Open topics (not covered):
  \begin{itemize}
    \item \texttt{move} keyword captures variables from the outside: 
    \item \href{https://doc.rust-lang.org/std/marker/trait.Sync.html}{Sync}, \href{https://doc.rust-lang.org/std/marker/trait.Send.html}{Send} marker traits
    \item Which stdlib fn is threadsafe? → indicated by traits
  \end{itemize}
\end{frame}

\begin{frame}[fragile]{threads}
  \begin{quote}
Simplifying things a bit, if the value is only read, it is captured by shared reference. If it is written, it is captured by mutable reference. If it is moved away from the closure, it is moved in.
  \end{quote}
\end{frame}

\begin{frame}[standout]
  async \& await
\end{frame}

\begin{frame}[fragile]{async \& await}
  \begin{minted}[fontsize=\scriptsize]{rust}
fn get_two_sites() {
  // Spawn two threads to do work.
  let thread_one = thread::spawn(|| download("https://foo.com"));
  let thread_two = thread::spawn(|| download("https://bar.com"));

  // Wait for both threads to complete.
  thread_one.join().expect("thread one panicked");
  thread_two.join().expect("thread two panicked");
}
  \end{minted}
  threading approach via \href{https://rust-lang.github.io/async-book/01_getting_started/02_why_async.html}{async-book}
\end{frame}

\begin{frame}[fragile]{async \& await}
  \begin{minted}[fontsize=\scriptsize]{rust}
async fn get_two_sites_async() {
  // Create two different "futures" which, when run to completion,
  // will asynchronously download the webpages.
  let future_one = download_async("https://foo.com");
  let future_two = download_async("https://bar.com");

  // Run both futures to completion at the same time.
  join!(future_one, future_two);
}
  \end{minted}
  async approach via \href{https://rust-lang.github.io/async-book/01_getting_started/02_why_async.html}{async-book}
\end{frame}

\begin{frame}[fragile]{async \& await}
  Similar to C\# and JavaScript.
  \mintinline{rust}{async} keyword. 

  \begin{minted}{rust}
async fn sub() -> u8 {
  42u8
}

fn main() {
  println!("{}", sub())
}
  \end{minted}
\end{frame}

\begin{frame}[fragile]{async \& await}
  \begin{minted}[fontsize=\scriptsize]{text}
meisterluk@gardner ~ % rustc --edition=2018 async1.rs

error[E0277]: `impl std::future::Future`
              doesn't implement `std::fmt::Display`
 --> async1.rs:6:20
  |
6 |     println!("{}", sub())
  |                    ^^^^^ `impl std::future::Future`
  |      cannot be formatted with the default formatter
  |
  = help: the trait `std::fmt::Display` is not implemented for
    `impl std::future::Future`
  = note: in format strings you may be able to use `{:?}`
    (or {:#?} for pretty-print) instead
  = note: required by `std::fmt::Display::fmt`
  = note: this error originates in a macro (in Nightly builds,
    run with -Z macro-backtrace for more info)

error: aborting due to previous error

For more information about this error, try `rustc --explain E0277`.
  \end{minted}
\end{frame}

\begin{frame}[fragile]{async \& await}
  \begin{quote}
  A \href{https://doc.rust-lang.org/std/future/trait.Future.html}{future} represents a value, that is not ready yet. Eventually, the future resolves to a value. \\
  ---\href{https://www.youtube.com/watch?v=skos4B5x7qE}{withoutboats} in a Rust Latam talk 2019
  \end{quote}
\end{frame}

\begin{frame}[fragile]{async \& await}
  \mintinline{rust}{await} keyword.
  Not \mintinline{rust}{await X}, but \mintinline{rust}{X.await}.

  \begin{minted}{rust}
async fn sub() -> u8 {
    42u8
}

fn main() {
    println!("{}", sub().await)
}
  \end{minted}
\end{frame}

\begin{frame}[fragile]{async \& await}
  \begin{minted}[fontsize=\scriptsize]{text}
meisterluk@gardner ~ % rustc --edition=2018 async2.rs

error[E0728]: `await` is only allowed inside `async`
              functions and blocks
 --> async2.rs:6:20
  |
5 | fn main() {
  |    ---- this is not `async`
6 |     println!("{}", sub().await)
  |                    ^^^^^^^^^^^ only allowed
  |         inside `async` functions and blocks

error: aborting due to previous error

For more information about this error, try `rustc --explain E0728`.
  \end{minted}
\end{frame}

\begin{frame}[fragile]{async \& await}
  So who can call the first async function? the \emph{executor} (scheduler)
  \begin{itemize}
    \item e.g. \href{https://crates.io/crates/async-std}{async-std}, \href{https://crates.io/crates/smol}{smol}, \href{https://crates.io/crates/tokio}{tokio}
    \item executor is thus exchangable
    \item executor allocates memory per future
    \item futures are like state machines between states
  \end{itemize}
  Components:
  \begin{itemize}
    \item Executor: is future X ready to go? Yes, then go. Else:
    \item Reactor: I will take care of the future
    \item Waker: Hey! We are ready, executor.
  \end{itemize}
\end{frame}

\begin{frame}[fragile]{async \& await}
  \begin{quote}
    The core method of future, poll, attempts to resolve the future into a final value. This method does not block if the value is not ready. Instead, the current task is scheduled to be woken up when it's possible to make further progress by polling again. The context passed to the poll method can provide a Waker, which is a handle for waking up the current task.
  \end{quote}
\end{frame}

\begin{frame}[fragile]{async \& await}
  \begin{minted}{rust}
use smol::io;
  
async fn sub() -> u8 {
  42u8
}


fn main() -> io::Result<()> {
    smol::run(async {
        println!("{}", sub().await);
        Ok(())
    })
    // prints 42
}
  \end{minted}
\end{frame}

\begin{frame}[fragile]{async \& await}
  \begin{minted}{rust}
async fn sub() -> u8 {
    42u8
}

fn main() {
    async {
        println!("{}", sub().await)
    };
}
  \end{minted}
\end{frame}

\begin{frame}[fragile]{async \& await}
  \begin{minted}[fontsize=\scriptsize]{text}
meisterluk@gardner ~ % rustc --edition=2018 async3.rs

warning: unused implementer of `std::future::Future`
         that must be used
 --> async3.rs:6:5
  |
6 | /     async {
7 | |         println!("{}", sub().await)
8 | |     };
  | |______^
  |
  = note: `#[warn(unused_must_use)]` on by default
  = note: futures do nothing unless you `.await`
          or poll them

warning: 1 warning emitted

meisterluk@gardner ~ % ./async3    # prints nothing
  \end{minted}
\end{frame}

\begin{frame}[fragile]{async \& await}
  Other async/await implementations:
  \begin{itemize}
    \item Calling an async function, schedules it
    \item Javascript calls \emph{Promise}, what rust calls \emph{Future}
    \item Green threads: a scheduler is compiled in every executable to manage small threads (M:N threading) (not OS threads!) (e.g. erlang, golang goroutines). rust 1.0 dropped green threads.
  \end{itemize}
  Rust implementations:
  \begin{itemize}
    \item Zero-cost: only await schedules the function (\enquote{lazy})
    \item Choose your own runtime
    \item Futures trait with \mintinline{rust}{poll(self: Pin<&mut Self>, cx: &mut Context)} \mintinline{rust}{-> Poll<Self::Output>}
  \end{itemize}
\end{frame}

\begin{frame}[fragile]{async \& await summary}
  \begin{itemize}
    \item Async-await is a way to write functions that can "pause", return control to the runtime, and then pick up from where they left off
    \item proposed 2016, didn't make edition 2018, landed in rust 1.39
    \item Memory management was a major problem
    \item \href{https://docs.rs/crate/wasm-bindgen-futures/0.4.17}{wasm-bindgen-futures} binds Rust Future to Javascript Promise
  \end{itemize}
\end{frame}

\begin{frame}[standout]
  On process-level
\end{frame}

\begin{frame}[fragile]{Process}
  \begin{minted}{rust}
use std::process::Command;

let output = Command::new("sh")
        .arg("-c")
        .arg("echo hello")
        .output()
        .expect("failed to execute process");

let hello = output.stdout;
  \end{minted}

  via \href{https://doc.rust-lang.org/std/process/struct.Command.html}{\mintinline{rust}{std::process::Command}}
\end{frame}

\begin{frame}[fragile]{Process}
  Open topics (not covered):

  \begin{itemize}
    \item Interprocess-communication (IPC)
    \item Shared memory
    \item Pipes
    \item Temporary files
    \item mmap
  \end{itemize}
\end{frame}

\begin{frame}[standout]
  But how many units?
\end{frame}

\begin{frame}[fragile]{How many concurrent units?}
  \textbf{How many?} Difficult.
  \begin{itemize}
    \item CPU cycles per instruction differs, caches everywhere!
    \item Thus, \textbf{benchmark, benchmark, benchmark!}
  \end{itemize}
  But how?
  \begin{itemize}
    \item No stable stdlib support in rust!
    \item \href{https://github.com/rust-lang/rust/issues/29553}{RFC 29553: Tracking issue for #[bench] and benchmarking support}
    \item e.g. \href{https://crates.io/crates/criterion}{criterion} crate
  \end{itemize}
\end{frame}

\begin{frame}[fragile]{How many concurrent units?}
  \textbf{Heuristics:}
  \begin{itemize}
    \item \href{https://github.com/rust-lang/rfcs/pull/985}{RFC 985: Add \mintinline{rust}{std::env::concurrency_hint}}
    \item \href{https://crates.io/crates/num_cpus}{num\_cpus} crate: \mintinline{rust}{num_cpus::get();}
    \item \texttt{dupfiles-go} and \href{https://github.com/BurntSushi/ripgrep/}{ripgrep} experience: the number of concurrent units reading files should roughly correspond to the number of logical CPUs
  \end{itemize}
\end{frame}

\section{Epilogue}


\begin{frame}[fragile]{Quiz}
  \begin{description}
    \item[What does SIMD stand for?] \hfill{} \\
      ~\uncover<2->{Single Instruction, Multiple Data}
    \item[What is a mutex?] \hfill{} \\
      ~\uncover<3->{a data structure to give mutually exclusive access to a code section}
    \item[What is a future?] \hfill{} \\
      ~\uncover<4->{represents a value, that is not ready yet}
    \item[Heuristically, what might be a good number of concurrent units?]
      ~\uncover<5->{number of logical CPUs of the machine}
  \end{description}
\end{frame}


\begin{frame}[fragile]{Next time}
  \begin{tabular}{ll}
    Next meetup  & Wed, 2020/08/26 \\
    Topic        & I/O (files, file formats, simple TCP server)
  \end{tabular}
\end{frame}

\begin{frame}[standout]
  Thank you!

  \includegraphics[width=40pt]{images/rustacean-flat-happy.png}
\end{frame}

\end{document}
